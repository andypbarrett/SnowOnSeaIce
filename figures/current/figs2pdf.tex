\documentclass{article}

\usepackage{graphicx}

\begin{document}

\begin{figure}[h!]
  \includegraphics[width=\textwidth, natwidth=528, natheight=529]{reanalysis_precip_figure01.png}
  \caption{Arctic Ocean domain with monthly mean locations for North Pole
    Drifting Stations}
\end{figure}

\newpage

\begin{figure}[h!]
  \includegraphics[width=1.5\textwidth, natwidth=720, natheight=504]{reanalysis_precip_figure02.png}
  \caption{Monthly climatologies of measured and corrected precipitation
    from North Pole Drifting Stations from Yang (1990) and Bogdanova et al
    (2002) show estimated precipitation is dependent on correction method.
    Average number of wet days (P > 1 mm) in each month are also shown.}
\end{figure}

\newpage

\begin{figure}[h!]
  \includegraphics[width=1.5\textwidth, natwidth=2160, natheight=1440, scale=2.]{reanalysis_precip_figure03.png}
  \caption{The current generation of global atmospheric reanalyses (MERRA2,
    ERA5, CFSRv2 and JRA55) span the operational lifetimes of both earlier
    and current altimetry missions.  However, earlier reanalyses do not
    cover planned lifetimes of CRYOSAT-2 or ICESat-2.}
\end{figure}

\newpage

\begin{figure}[h!]
  \includegraphics[width=\textwidth, natwidth=528, natheight=529]{reanalysis_precip_figure04.png}
  \caption{Mean precipitation rates from global reanalyses included in this
    study for the globe and Polar Cap.  Global precipitation rates reveal
    artifacts introduced by changes in observing systems (e.g. MERRA, see
    Bosilovich et al, 2017) but these artifacts are not apparent in
    precipitation rates for the Polar Cap.}
\end{figure}

\newpage

\begin{figure}[h!]
  \includegraphics[width=1.7\textwidth, natwidth=1500, natheight=800]{accumulation_precip_stats_for_npsnow_region.png}
  \caption{Arctic Ocean mean total precipitation for August to April
    snow on sea ice accumulation period from reanalyses vary by about
    100 mm.  CFSR and MERRA2 produce the most precipitation.  ERA-Interim
    and MERRA have the least precipitation.  ERA5 and JRA55 have very
    similar precipitation amounts.  MERRA2 appears to converge with these two
    ``middle of the road'' estimates.  All reanalyses show good temporal
    correlation.  In particular, all reanalyses exhibit high precipitation in
    1990, 1995 and 2017.  Partitioning total precipitation into drizzle and
    light/heavy precipitation provides insight into the causes of these
    differences.  CFSR produces more light/heavy precipitation than other
    reanalyses but similar volumes of drizzle.  MERRA2 on the other hand
    produces both more drizzle and light/heavy precipitation.  JRA55
    produces the least volume of drizzle of all reanalyses.}
\end{figure}

\newpage

\begin{figure}[h!]
  \includegraphics[width=\textwidth, natwidth=1000, natheight=1000]{arctic_ocean_daily_precipitation_cdf_annual.png}
  \caption{Between 64\% and 82\% of daily precipitation totals from reanalyses
    in the Arctic Ocean are less than 1 mm.  Daily precipitation less than 1 mm
    is often considered as drizzle (e.g. Sun et al, 2006).  Probability
    distributions for non-zero daily precipitation from North Pole drifting
    stations indicate as much as 90\% of daily total precipitation is less
    than 1mm.  CDFs for reanalysis precipitation are generated by binning
    daily precipitation for each day in the 1980-01-01 to 2015-12-31 period
    for each 50 km grid cell within the Arctic Ocean domain.}
\end{figure}

\newpage

\begin{figure}[h!]
  \includegraphics[width=1.7\textwidth, natwidth=850, natheight=500]{reanalysis_precip_figure07.png}
  \caption{Spatial patterns of August to April accumulation period total
    precipitation are broadly similar for all reanalyses with a drier central
    Arctic Ocean and a wetter North Atlantic region extending into Barents
    and Kara seas.  CFSR and MERRA2 stand out showing a tendency for more
    precipitation over the eastern Arctic Ocean.  Note also that CFSR appears
    to have a noisy speckled patterns of precipitation over northern
    North America and Eurasian land masses.}
\end{figure}

\newpage

\begin{figure}[h!]
  \includegraphics[width=1.7\textwidth, natwidth=850, natheight=500]{reanalysis_precip_figure08.png}
  \caption{Partitioning August to April precipitation into light/heavy
    precipitation and drizzle reveals that all reanalyses have very
    similar patterns in mean daily precipitation grteater than 1 mm.}
\end{figure}

\newpage

\begin{figure}[h!]
  \includegraphics[width=1.7\textwidth, natwidth=850, natheight=500]{reanalysis_precip_figure09.png}
  \caption{Although patterns of August to April mean light/heavy precipitation
    are similar, frequency of wet days for MERRA, ERA-Interim, JRA55 and
    ERA5, constrast with CFSR and MERRA2, which show higher frequency of
    wet days in the eastern Arctic.}
\end{figure}

\newpage

\begin{figure}[h!]
  \includegraphics[width=1.7\textwidth, natwidth=850, natheight=500]{arctic_precipitation.accumulation_period.climatology.drizzle.cfsr_totprec.png}
  \caption{Differences between MERRA2 and other reanalyses can be seen in
    spatial patterns of drizzle.  MERRA2 has a large amount of drizzle over
    the northern CAA, the New Siberian Islands, the Laptev Sea coast, and,
    to a lesser extent, Franz Josef Land. MERRA2 also has higher amounts of
    drizzle over the central Arctic Ocean than the other reanalyses.}
\end{figure}

\newpage

\begin{figure}[h!]
  \includegraphics[width=1.7\textwidth, natwidth=1000, natheight=600]{annual_precipitation_reanalysis_with_obs.png}
  \caption{Observations of precipitation over the Arctic Ocean are limited to North Pole Drifting stations.  Observations overlap with reanalyses for the 1979 to 1991 period.  Biases in reanalysis precipitation depend on how precipitation measurements are corrected.  Comparisons with Bogdanova et al (2002) indicate all reanalyses have too much precipitation.  However, comparisons with Yang (1990) MERRA, ERA-Interim, ERA5 and JRA55 produce similar annual totals of precipitation, at least to the period of overlap.  Vertical bars to right of plot show the ranges of observations and reanalysis annual total precipitation for the 1970 to 1990 period.  Dots show mean annual total precipitation for this period.}
\end{figure}

\newpage

\begin{figure}[h!]
  \includegraphics[width=1.7\textwidth, natwidth=1000, natheight=800]{yang_trajectory_reanalysis_bias.png}
  \caption{Although reanalyses compare well corrected precipitation from Yang (1990), bias vary seasonally.  In general, during the accumulation period, reanalyses are lower than observations.  The exception is MERRA2, which for most months is higher than obervations.  During the May to August period, months with high percipitation in the central Arctic, reanalyses over-estimate precipitation}
\end{figure}

\newpage

\begin{figure}[ht]
  \includegraphics[width=1.5\textwidth, natwidth=1000, natheight=1131]{arctic_ocean_daily_precipitation_cdf_month.png}
  \caption{\textbf{ADDITIONAL FIGURE} Drizzle tends to make up more of precipitation in winter months.}
\end{figure}

\newpage

\begin{figure}[ht]
  \includegraphics[width=1.5\textwidth, natwidth=1500, natheight=800]{reanalysis_arctic_mean_drizzle.Nh50km.png}
  \caption{\textbf{ADDITIONAL FIGURE} For the Arctic as a whole, MERRA2 and MERRA have the largest amount of drizzle.  CFSR, ERA-Interim and ERA5 cluster together with 50 to 60 mm of drizzle.  JRA55 has the smallest amount of drizzle.}
\end{figure}

\begin{figure}[ht]
  \includegraphics[width=1.5\textwidth, natwidth=1500, natheight=1000]{arctic_regional_precip_cdf.png}
  \caption{\textbf{ADDITIONAL FIGURE} Regions located at the ends of the Atlantic and Pacific storm tracks (Greenland, Barents and Bering) have the smallest amount of drizzle.  Beaufort and Central Arctic have the highest (CHECK!)}
\end{figure}

\newpage

\begin{figure}[h!]
  \includegraphics[width=1.5\textwidth, natwidth=1500, natheight=800]{reanalysis_precip_figure11.png}
  \caption{\textbf{ADDITIONAL FIGURE} Larges differences between reanalyses can be seen for Arctic ocean mean wetday frequency.}
\end{figure}

\end{document}
